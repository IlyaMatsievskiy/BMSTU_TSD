\documentclass[12pt, a4paper]{article}
\usepackage[utf8]{inputenc}
\usepackage[T2A]{fontenc}
\usepackage[english,russian]{babel}
\usepackage{ragged2e}
\usepackage{amsmath}
\usepackage{tikz}
\usepackage{setspace}
\usepackage{hyperref}
\usepackage{mathtext}
\linespread{1.3}
\renewcommand{\figurename}{Рис.}
\usepackage[left=20mm, top=15mm, right=15mm, bottom=20mm]{geometry}


\begin{document}
\renewcommand{\labelenumii}{\arabic{enumi}.\arabic{enumii}}
\begin{titlepage}
    \begin{center}
	МГТУ им. Н. Э. Баумана\\black
	Факультет ФН <<Фундаментальные Науки>>\\
	Кафедра ФН-12 <<Математичсекое моделирование>>
	
	\vspace{8cm}
	Отчет по учебной практике\\
	<<Использование компьютерной верстки \LaTeX>>\\
    ``Вариант №7''
\end{center}
\begin{flushright}
	\vspace{10cm}
	Студент: Мациевский И. М. \\
	Преподаватель: Юрченков А. В.\\
	Группа: ФН12-31Б
\end{flushright}

\begin{center}
	\vspace{1cm}
	Москва 2023
\end{center}
\end{titlepage}

\newpage
\begin{center}
    \section*{Задание 1}
\end{center}
\justifying
Для соответсвующего уравнения кривой необходимо:
\par  1.1 записать каноничекое уравнение;
\par  1.2 определить тип кривой;
\par  1.3 записать преобразование параллельного переноса, приводящее уравнение к канони-ческому виду; 
\par 1.4 найти в случае эллипса: полуоси, эксцентриситет, координаты центра, вершин и фокусов; в случае гиперболы: полуоси, эксцентриситет, координаты центра, вершин и фокусов, уравения асимптот; в случае параболы: параметр, вершину, фокус, уравнение директрисы (все координаты и уравнения - в исходной системе координат);
\par  1.5 сделать чертеж кривой в исходной системе координат средствами \TeX.
\par \textbf{Замечание.} Для каждого из пунктов 1.3-1.5 необходимо сделать отдельный чертеж.

\subsection*{1.1}
Дано уравнение $y^2+3x+4y=2$ кривой второго порядка. Приведем уравнение к каноническому виду.
\begin{center}
     $(y^2+4y+4)-4+3x=2$\\
     $(y+2)^2=6-3x$\\
     $(y-(-2))^2=2*(-1.5)*(x-2)$\\
     Введем замену: $\bar x = x-2; \bar y = y+2$\\
     $\bar y^2 = -3\bar x $ - каноническое уравнение\\
\end{center}

\subsection*{1.2}
Мы получили уравнение параболы.
\newpage
\subsection*{1.3}
Запишем преобразование параллельного переноса, приводящее уравнение к каноническому виду:
\begin{equation*}
 \begin{cases}
   \bar x = x - 2\\
   \bar y = y + 2\\
 \end{cases}
\end{equation*}
\begin{figure}[h!]
    \begin{center}
      \begin{tikzpicture}
      \draw[->] (-5,0) -- (5,0) node[right] {$x$};
      \draw[->] (0,-5) -- (0,5) node[above] {$y$};
      \fill[black] (0,0) circle (3pt);
      \node[below right] at (0,0) {O};
      \draw[->,black](-5,-2)--(5,-2) node[right, text=black]{$\bar x$};	
      \draw[->,black](2,-5)--(2,5) node[above]{$\bar y$};
      \fill[black] (2,-2) circle (2pt);
      \node[below left] at (2,-2) {$\bar O$ (2,-2)};  
      \foreach \x in {-4,-3,-2,-1,1,2,3,4}
      \draw (\x,0.1) -- (\x,-0.1);
      \foreach \y in {-4,-3,-2,-1,1,2,3,4}
      \draw (0.1,\y) -- (-0.1,\y);
      \node[below] at (1,0) {1}; 
      \node[below] at (2,0) {2}; 
      \node[left] at (0,1) {1};
      \node[left] at (0,-2) {-2};
    \end{tikzpicture}
    \caption{Преобразование параллельного переноса}
    \end{center}
\end{figure}
\newpage

\subsection*{1.4}
Из канонического уравнения найдем параметр, вершину
и фокус параболы.\\
\begin{center}
	$(y-(-2))^2=2*(-1.5)*(x-2)$
\end{center}
\begin{itemize}
    \item Значит параметр p = -1.5;
    \item Координата вершины - $\bar O(2,-2)$;
    \item Фокус - $(\frac{p}{2},y_{ver})$,  то есть 
    $(-\frac{3}{4}, -2)$.
\end{itemize}

\begin{center}
\begin{tikzpicture}[domain=-6:2,scale=0.7]
  \fill[black] (0,0) circle (3pt);
  \node[below right] at (0,0) {O};
  \draw[->] (-6,0) -- (6,0) node[right] {$x$};
  \draw[->] (0,-6) -- (0,6) node[above] {$y$};
  \draw[color=black]  plot (\x, {sqrt(6- \x * 3)- 2}) {};
  \draw[color=black]  plot (\x, {-sqrt(6- \x * 3)- 2}) {};
  \foreach \x in {-5,-4,-3,-2,-1,1,2,3,4,5}
  \draw (\x,0.1) -- (\x,-0.1);
  \foreach \y in {-5, -4,-3,-2,-1,1,2,3,4,5}
  \draw (0.1,\y) -- (-0.1,\y);
  \fill[black] (-0.75, -2) circle (2pt);
  \node[below left] at (-0.75,-2) {F};
  \node[below] at (1,0) {1}; 
  \node[below] at (2,0) {2}; 
  \node[right] at (0,-2) {-2};
  \node[left] at (0,1) {1};
  \fill[black] (2,-2) circle (2pt);
  \node[below right] at (2,-2) {$\bar O$ (2,-2)}; 
\end{tikzpicture}
\end{center}

\subsection*{1.5}
Изобразим итоговый результат: парабола, ее вершина, фокус и преобразование параллельного переноса\\
\begin{figure}
\centering
\begin{tikzpicture}[domain=-6:2,scale=0.7]
  \fill[black] (0,0) circle (3pt);
  \node[below right] at (0,0) {O};
  \draw[->] (-6,0) -- (6,0) node[right] {$x$};
  \draw[->] (0,-6) -- (0,6) node[above] {$y$};
  \draw[color=black]  plot (\x, {sqrt(6- \x * 3)- 2}) node[right] {};
  \draw[color=black]  plot (\x, {-sqrt(6- \x * 3)- 2}) {};
  \foreach \x in {-5,-4,-3,-2,-1,1,2,3,4,5}
  \draw (\x,0.1) -- (\x,-0.1);
  \foreach \y in {-5, -4,-3,-2,-1,1,2,3,4,5}
  \draw (0.1,\y) -- (-0.1,\y);
  \fill[black] (-0.75, -2) circle (2pt);
  \node[below left] at (-0.75,-2) {F};
  \node[below] at (1,0) {1}; 
  \node[below] at (2,0) {2}; 
  \node[right] at (0,-2) {-2};
  \node[left] at (0,1) {1};
  \fill[black] (2,-2) circle (2pt);
  \node[below right] at (2,-2) {$\bar O$ (2,-2)};
  \draw[->,black](-5,-2)--(5,-2);	
  \draw[->,black](2,-5)--(2,5); 
\end{tikzpicture}
\caption{Парабола и ее фокус}
\end{figure}
\newpage
\section*{Задание 2}
Построить кривую средствами \TeX, вводя соответствующую систему координат.\\
Дано уравнение:\\
$x=2+\sqrt{4-2y}$\\
Можно заметить, что это уравнение описывает правую 
ветвь параболы.\\
$x-2=\sqrt{4-2y}$\\
Возведем в квадрат и приведем к каноническому виду,
чтобы удобнее было выполнить замену, не забудем про 
ограничение $x-2 \geq 0 \Rightarrow x \geq 2$\\
$(x-2)^2=4-2y \Rightarrow (x-2)^2=-2*(y-2)$\\
Выполним замену: $\bar x = x-2; \bar y = y-2$\\
$\bar x^2=-2\bar y$\\
Построим правую ветвь этой параболы в двух 
системах координат:\\
(x, y) - первый график: вершина лежит в точке $\bar 
O(2,2)$, фокус F(2, -3)\\
$(\bar x, \bar y)$ - второй график: вершина лежит в
точке $\bar O(0, 0)$, фокус F(0,-1)
\begin{figure}
\begin{center}
\begin{tikzpicture}[domain=2:6,scale=0.7]
  \fill[black] (0,0) circle (3pt);
  \node[below right] at (0,0) {O};
  \draw[->] (-6,0) -- (6,0) node[right] {$x$};
  \draw[->] (0,-6) -- (0,6) node[above] {$y$};
  \draw[color=black]  plot (\x, {-2-(\x - 2)^2 / 2}){};
  \foreach \x in {-5,-4,-3,-2,-1,1,2,3,4,5}
  \draw (\x,0.1) -- (\x,-0.1);
  \foreach \y in {-5, -4,-3,-2,-1,1,2,3,4,5}
  \draw (0.1,\y) -- (-0.1,\y);
  \fill[black] (2, -3) circle (2pt);
  \node[below left] at (2,-3) {F};
  \node[below] at (1,0) {1}; 
  \node[below] at (2,0) {2}; 
  \node[right] at (0,-2) {-2};
  \node[left] at (0,1) {1};
  \fill[black] (2,-2) circle (2pt);
  \node[above right] at (2,-2) {$\bar O$ (2,-2)};
\end{tikzpicture}
\end{center}
\caption{Правая ветвь параболы в координатах по (x,y)}
\end{figure}

\begin{figure}
\begin{center}
\begin{tikzpicture}[domain=0:4,scale=0.7]
  \fill[black] (0,0) circle (3pt);
  \node[below right] at (0,0) {O};
  \draw[->] (-6,0) -- (6,0) node[right] {$\bar x$};
  \draw[->] (0,-6) -- (0,6) node[above] {$\bar y$};
  \draw[color=black]  plot (\x, {-\x^2 / 2}) node[right] {};
  \foreach \x in {-5,-4,-3,-2,-1,1,2,3,4,5}
  \draw (\x,0.1) -- (\x,-0.1);
  \foreach \y in {-5, -4,-3,-2,-1,1,2,3,4,5}
  \draw (0.1,\y) -- (-0.1,\y);
  \fill[black] (0, -1) circle (2pt);
  \node[below left] at (0,-1) {F};
  \node[below] at (1,0) {1}; 
  \node[left] at (0,1) {1};
  \fill[black] (0,0) circle (2pt);
  \node[above right] at (0,0) {$\bar O$ (0,0)};
\end{tikzpicture}
\end{center}
\caption{Правая ветвь параболы в координатах по $(\bar x,\bar y)$}
\end{figure}

\newpage
\begin{center}
    \section*{Задание 3}
\end{center}
\justifying
Построить кривую средствами \TeX, указав соответствующую систему координат.\\
Оси симметрии эллипса параллельны осям координат OX
и OY, А(3,1) - вершина эллипса, $F_1(-1,4)$ - его 
фокус.\\
Отметим вершину и фокус на координатной плоскости, 
фокус лежит на большой оси, она параллельна оси OX, 
проведем ее, она задается уравнением $y=4$. Вершина 
не лежит на проведенной оси, следовательно она лежит 
на малой оси, она будет задаваться уравнением $x=3$.
\\
Каноническое уравнение эллипса : $\frac{(x-h)^2}
{a^2}+\frac{(y-k)^2}{b^2}=1$. Из него можно найти 
координаты фокусов: $F_1(h-c, k)$, $F_2(h+c, k)$,
отсюда:\\
\begin{equation*}
 \begin{cases}
   h - c = -1\\
   h+c=7
 \end{cases}
 \Rightarrow
  \begin{cases}
   h=3\\
   c=4
 \end{cases}
\end{equation*}
Найдем вершину B, расположенную на малой оси 
эллипса, она симметрична вершине А относительно
большой оси, значит $B(3,7)$.\\
Координаты вершин: $A(h, k-b)$, $B(h, k+b)$,
отсюда:\\
\begin{equation*}
 \begin{cases}
   k-b=1\\
   k+b=7
 \end{cases}
 \Rightarrow
  \begin{cases}
   k=4\\
   b=3
 \end{cases}
\end{equation*}
Найдем оставшийся коэффициент $a$ из формулы: $a^2=
b^2+c^2$: $a^2=9+16=25$, а значит каноническое 
уравнение искомого эллипса имеет вид:\\
\begin{center}
	$\frac{(x-3)^2}
{25}+\frac{(y-4)^2}{9}=1$
\end{center}
\newpage
\begin{figure}
\begin{tikzpicture}
  \draw[dashed] (3, 8) -- (3, -1); 
  \draw[dashed] (-2, 4) -- (8, 4); 
  \draw[thick] (3, 4) ellipse (5 and 3);
  \draw[->] (-2, 0) -- (8, 0) node[right] {$x$};
  \draw[->] (0, -1) -- (0, 9) node[above] {$y$};
  \foreach \x in {-2, -1, 1, 2, 3, 4, 5, 6, 7, 8}
  \draw (\x, 0.1) -- (\x, -0.1) node[below] {$\x$};
  \foreach \y in {-1, 1, 2, 3, 5, 6, 7, 8}
  \draw (0.1, \y) -- (-0.1, \y) node[left] {$\y$};
  \node at (3, 4) {$\bullet$};
  \node[below right] at (3, 4) {$(3, 4)$};
  \node at (5, 4) [above left] {$a=5$};
  \node at (3, 7) [below right] {$b=3$};
  \node[below left] at (0,0) {$0$};
  \node[below left] at (3,1) {$A(3,1)$};
  \node[below left] at (3,7) {$B(3,7)$};
  \node[below right] at (-1,4) {$F_1(-1,4)$};
  \node[below right] at (7,4) {$F_2(7,4)$};
  \fill[black] (3, 1) circle (2pt);
  \fill[black] (-1, 4) circle (2pt);
  \fill[black] (7, 4) circle (2pt);
  \fill[black] (3, 7) circle (2pt);
\end{tikzpicture}
\caption{Заданный эллипс}
\end{figure}
\newpage
\begin{center}
\begin{thebibliography}{}
\addcontentsline{toc}{section}{Список литературы}
\bibitem{geom} А. Н. Канатников, А. П. Крищенко : Аналитическая геометрия. -- 2-е изд., Москва, 2000 г. Издание МГТУ им. Н. Э. Баумана.
\bibitem{geom} Till Tantau : The TikZ and PGF Packages Manual. --  for version 1.18, Institut fur Theoretische Informatik Universitat zu Lubeck. June 12, 2007
\bibitem{geom} С. М. Львовский : Набор и вёрстка в 
системе \LaTeX. 3-е издание исправленное и 
дополненное, 2003 год.
\end{thebibliography}
\end{center}
\end{document}